Special thanks to The John S. and James L. Knight Foundation, The Tow Foundation
and the team from the Tow Center for Digital Journalism at Columbia University,
including Director Emily Bell, Research Director Taylor Owen and Research
Associate Lauren Mack. Additional thanks to our Research Associate, Kevin Matha.
Our gratitude to Abigail Ronck for copy editing this Tow/Knight brief. Without all
of your support this project would not have been possible. We also wish to thank our
esteemed study participants, as well as our panelists, for striving to improve the news
on a daily basis.

\chapter{I.Introduction: Studying the Single-Subject News Model}
The emerging trend of the single-subject news website is a media byproduct
of our times—the response to a fundamental shift in content supply
and demand. Traditional newsrooms, under competitive and commercial
pressures, have been unable to maintain consistent coverage of complex
stories. As a result, the supply of in-depth coverage or sustained reporting
on certain issues has dropped. In parallel, the Internet has created a space
for new communities of interest, surfacing a micro-audience around many
of the same issues that have disappeared from the mainstream press. The
result has been an opportunity for a change in format to a smaller-scale
news delivery paradigm for topic-specific reporting delivered online to a
dedicated digital audience.
The shift toward what we’ve dubbed ``hypertopical'' news represents the
activation of the ``long tail'' of journalism, or a shift away from the center
of the bell curve in news coverage (the center being stories and issues that
have maximum broad-based appeal). At the tails of that content bell curve
lie stories and issues that appeal to a more limited audience. The tails are
where one finds niche news markets with a latent demand for reporting
topics that don’t make it onto the evening news.

The age of digital journalism has made it more economically feasible to
serve those markets. It has eliminated the barriers to entry and lowered
the cost of production for new publishers, while raising the prospects of
discovery for a new audience. Today’s digital reporters can serve a loyal and
geographically dispersed community of interest.
The agent of change is the entrepreneurial journalist, the next generation
publisher who becomes the de facto founder of a small media enterprise.
In some cases the founding publisher is a career reporter; in other cases a
subject-matter expert with profound knowledge of a particular domain. The
element of focus—the fusing of deep domain knowledge with the elements
of journalism—can have an enriching effect on the end product.
For this study, conducted at the Tow Center for Digital Journalism, we collected
20 such examples: practitioners of the single-subject news website
in domains varying from foreign affairs (e.g., Tehran Bureau, North Korea
News) to science (e.g, Deep-Sea News, HealthMap) to domestic and local
policy issues (e.g., Education News Network/Chalkbeat, Gotham Gazette).
Each is diverse in its structure and methodology, with operations ranging
in size from one reporter to more than 50. Their founders likewise vary in
age and experience, from veteran journalists to muckraking graduate students.
Some are foundation-backed, others are self-funded, and almost all
are exploring alternative revenue streams.
Still, the sample collection presented a consistent set of results: The focus
on a niche topic captures a smaller audience than mainstream news outlets
(which cover a broad range of topics), but with a higher loyalty and intensity
of engagement within that audience as measured by user return rates. Each
case study also revealed overlapping challenges, from finding sustainable
business models, to deploying technology and digital marketing, to managing
staff turnover. These challenges represent the immensity of the task as
journalists-turned-entrepreneurs come to fuse two distinct skill sets: the
craft of professional news reporting and the practice of running an iterative
startup culture.


Based on initial interviews with our 20 selected publishers, we compiled
preliminary findings into this research paper; at the end of our study we
intend to publish a more comprehensive report. In this Tow/Knight brief,
we will outline the exigencies of the model and present the highlights of
input from our research participants. Over time we will look to refine our
suggested parameters for the model, compile and examine best practices,
and suggest norms and guidelines for single-subject publishers that remain
consistent with traditional journalistic values.

\chapter{II. Overview: The High Hopes and Potential Pitfalls of a Trend}
Over the past decade the single-subject news model has risen in prestige
and legitimacy within the media establishment. A recent crowning moment
was the awarding of a 2013 Pulitzer Prize to InsideClimate News, a single-
subject website covering environmental issues. With a combination of
consistent investigative reporting and scientific expertise, its journalists
covered a previously underreported ecological disaster on the Kalamazoo
River in Michigan in a report that revealed the pervasive weakness of pipelines
carrying heavy oil products into the United States.
The success of InsideClimate News represents the peak upside of singlesubject
news outlets. The potential benefits are high: single-subject websites
can dramatically raise the supply of high-quality journalism, covering complex
and chronic issues that are widely neglected in mainstream media. To
the publisher, the niche news model represents an unprecedented opportunity
to serve a hyper-focused audience, capturing the market and building
a community among return users. To the beat reporter, niche news outlets
can represent a return to public service journalism, fact-based and in-depth
work of the highest caliber. In short, the sites provide an opportunity for
focus in an age where mainstream newsrooms can seldom afford the luxury
of consistently covering a niche issue.


But there are also downsides and difficulties buried within the trend. Even
with a lower cost base, startup news outlets struggle with the financial viability
of digital news. Outside of foundation funding, consistent revenue
streams are hard to find, although some of our participants have taken innovative
steps toward business development. With business models themselves
in a state of creative flux, the boundaries for ethical fundraising are
being tested as publishers look to uphold objective and neutral journalism.
In the interim, many single-subject news operations are being sustained and
self-financed by a passionate founding team and often a dedicated corps of
volunteers—an admirable but rarely sustainable underpinning.
There are potential pitfalls implicit in this media fragmentation. The credibility
of any given news report has always rested on a two-part formula:
the credibility of the journalist, as well as that of the news outlet. In the
emerging single-subject model, startup outlets lack the institutional guarantee
and practical spot check that a major media outlet provides. Consistency
and continuity are less assured as news startups build sustainable
financing from scratch. For some publishers there may be less of an incentive
to maintain neutrality, as niche news sites can find support among an
impassioned and often opinionated audience bound by groupthink toward
a particular agenda.
The counterpoint to that downside risk has been brand building at the
startup level, as seen in the earned credibility of digital publications like
SCOTUSblog and Homicide Watch, which later evolved into sister sites
Homicide Watch Chicago and Homicide Watch Trenton. Since their earliest
days, these online news outlets have made a significant contribution to
the knowledge pool, adding heft to the media landscape within their topical

domains. Their success represents the gradual growing up of news startups.
A previous Tow Center publication called it ``the institutional rebirth''1 of
the news industry, through mature digital news outlets.
That leaves open a broader question of systemic risk to the media ecosystem.
If the in-depth reporting of whole topics is left to individual news sites,
then the media system becomes more reliant on those niche providers. The
mainstream press may functionally outsource whole topics to niche providers
outside the boundaries of their oversight and accountability. The system
would then narrow its supply chain and limit its ability to cross-check information;
as a result it would become more difficult to discern whether that
niche outlet, and by extension the system at large, is providing complete
and unbiased coverage of an issue. We do not yet see this taking place, but
we do see manifestations of the sequence at an early stage, as single-subject
websites become the go-to source for a multitude of bigger news outlets
borrowing their coverage.
For all those reasons—for their upside potential and downside risk—single-
subject news websites are a critical field of study. These startup outlets
will proliferate and pepper the media landscape. They are already reshaping
the information ecosystem, adding depth and addressing the deficits in the
mainstream media market. They need to be well understood and placed
within the universe of journalism as they become a norm in the digital age.
1 C.W. Anderson, Emily Bell, and Clay Shirky. Post-Industrial Journalism: Adapting to the Present
(Tow Center for Digital Journalism, Columbia University, 2012): p. 46.


\chapter{III. Defining the Scope}
The first step in studying the single-subject news model was to define
it. What is it? Who runs it? How do we validate any given example?
What separates a blog from what should be considered a bona fide digital
news source?
For the sake of this study we set interim parameters on the model, a defined
scope for how we identified 20 emerging examples. The detailed criteria are
stated in our published scope of study on the Tow Center website.2 In synopsis,
a single-subject news outlet as defined in our research requires:
1. In-Depth: A website must address one topic in depth, delving
into a single story with a single angle within a broader story; it
brings a narrow focus on a topic perceived by the founder to be
underreported and underserved within mainstream sources.
2. Narrow in Scope: The topic addressed must be sufficiently narrow
in scope. For example, a website that deals with ``U.S. News''
as a topic is not a single subject website but one that deals with
``U.S. Healthcare'' on an in-depth level is because it addresses
2 Kristin Nolan and Lara Setrakian, ``The Rise of the Single-Subject Platform,'' Tow Center for Digital
Journalism. Tow Center for Digital Journalism, Columbia University, 28 Aug. 2013. Web. 28 Aug. 2013.

a niche audience within a larger topic, filling a gap in available
information about that issue. Often, the test of a niche news site
will come in the website’s audience; the target audience, or user
base, will reflect the relatively narrow focus of the reporting.
3. Fact Based: The website must feature a clear emphasis on factbased
reporting versus opinion. For instance, some blogs may
have similar characteristics to a single-subject website in topical
focus, however, if they are overwhelmingly based on opinion,
rather than reporting the facts, then we would not consider
them eligible for this study. Similarly, ``conspiracy theory'' websites
aimed at promoting or debunking a particular point of view
would not be considered in this study.
4. Origins Online: The organization must have its origins online
rather than converted from a prior publication or existing news
media outlet. The rationale behind this qualifier is to highlight
the rising trend of entrepreneurial websites designed to fill a
gap, not a pre-existing news provider that transitioned to the
online market.
5. Independently Funded: The website must be funded by
non-governmental sources by private actors (i.e. nongovernmental
entities).
6. Exemption: The focus must be differentiated from that of a
hyperlocal city newspaper. City newspapers, while niche in
focus, are not eligible for this study. While many city newspapers
conduct in-depth investigative research on issues of local
importance, this coverage tends not to go to the national level,


and therefore does not fill a gap in mainstream media, although
there are clear exceptions of cases where a local story has
gone national.
7. Exemption: We also limited our scope to English-language publishers
as an expediency measure. While there are many concrete
examples of non-English sources who do in-depth investigative
journalism, due to the language constraints of the Research
Team, these sources are unable to be studied at this time.
Within these boundaries there is a great diversity of models and approaches
to executing the single-subject mandate. Yet there is a consistency to their
experience, along with shared challenges related to the focus and format of
their journalistic endeavors.

\chapter{IV. Common Cause: Shared Motives and Execution Strategies}
Single-subject websites aim to fix a perceived flaw or deficit in the news
cycle, providing coverage of underreported issues within their domains.
Their publishers are generally moved by a perceived market failure, or an
understanding that the mainstream media is not providing adequate coverage
of a given topic.
Such was the motivation to start InsideClimate News, according to its
founder, David Sassoon:
Media coverage of the climate issue has been an enormous failure.
Look at the significant number of people in this country who do not
understand it and are misinformed about it.
A lot of it stems from the confusion in the media, and the awkward
way the issues were being covered. The media was reporting on the
science inaccurately. The scientific community was 90, 95 percent
certain about the human contribution to climate change. You could
see the progression of certainty through the IPCC reports, yet the
media allowed political operatives to confuse the science.


Sassoon and other publishers in our study built their models based on subject
matter mastery, with writers who have a focused specialty and often
advanced knowledge of that topical domain. In some cases those writers
were career journalists, in others they were subject matter specialists:
North Korea News was founded by a former U.N. staffer; Deep-Sea News by
an ocean researcher with a Ph.D.; Bleacher Report by a group of friends who
wanted better news coverage of their favorite sports teams. In each case, a
depth of knowledge served to boost their reporting skills and amplify their
credibility within a given topic community.
Their specialized knowledge also made it possible for them to harness new
kinds of information, previously untapped by mainstream reporters. Our
sole study participant from the field of data journalism, HealthMap collects
data and open source information on the spread of infectious diseases
and delivers reporting that can better inform public health professionals.
Founded by John Brownstein, an epidemiologist, and Clark Freifeld, a computer
scientist, the site can functionally capture public health signals from
social media, harnessing information into data reports.
``The idea was that there’s a lot of information being posted online via the
courts or from mailing lists or blogs or Twitter or Facebook,'' said Anna
Tomasulo, a Program Coordinator for HealthMap, ``that really gives signals
of outbreaks far earlier than traditional public health reporting is
able to do.''
Nearly all of the publishers in our study used a strategy of active differentiation:
their reporting consciously aimed to cover a niche that is left absent
by the mainstream press. One example is StartUp Beat, a news outlet that
covers ``the world’s most innovative early-stage startups,'' with a specific
focus on entrepreneurs outside the Silicon Valley spotlight. Founder Brian
Kovalesky differentiates his news coverage by deliberately seeking startups
that are harder to find in the more mature digital technology press,
specifically benchmarking against publications like Mashable, TechCrunch
and BusinessInsider.

``What I noticed with these outlets were that they were still an insider’s
game…entrepreneurs that aren’t plugged into the scene are ignored,''
said Kovalesky.
Similarly, Education News Network (Chalkbeat) deliberately focuses
on covering schools in low-income communities, which, in the words of
founder, Elizabeth Green, are ``traditionally news deserts.'' IA Reporter,
which covers international arbitration, looks at ``high-stakes international
lawsuits that are just sort of falling in the cracks and are not covered by
the legal and financial press,'' according to its founder, Luke Peterson. The
Gotham Gazette, a news site focused on public policy in New York, deliberately
reduced the sub-topics it covers from 21 to eight, eliminating categories
like education and arts policy. It was a strategic choice to scale back in
areas where competing news outlets were growing.


\chapter{V. Contrasting Models: Target Audience, Structural Setups and Major Media Ties}
While single-subject websites all serve a defined community of interest,
participants in our study conveyed a range of strategies for identifying and
engaging their target audience within a given niche. Some are aimed at an
elite readership of subject specialists. Archinect and IA Reporter mainly
serve an audience of dedicated professionals; Archinect targets practicing
architects and academics while IA Reporter focuses on legal professionals
interested in international arbitration. OpenCanada, a website devoted to
Canadian foreign policy, serves an elite national readership interested in
global affairs. That scope consciously limits the size of its audience.
``It’s a double niche,'' said founder, Taylor Owen. (Owen is also the Director
of Research at Columbia University.) ``Instead of just the one niche of foreign
policy, we add another filter to that, which is Canada.''
In a hybrid model, HealthMap is targeting a mix of medical professionals
(as information consumers) and the general public (as information providers).
The model relies, in part, on self-reporting by citizens, who contribute
data on the spread of infections in their vicinity. In a similar mode of thinking,
SoccerDrugs (Fussball Doping), a website focused on doping in professional
soccer, has a target audience of sports fans and sports officials—both
the consumers and sources of information.

Other websites are explicitly aiming for a general audience with a particular
interest, such as Deep-Sea News and FactCheck.org. Within their respective
domains—oceans news and political accountability—each platform
was created to serve a broad public with accessible insights in its field. Craig
McClain, the founder of Deep-Sea News, describes it as a strategy of ``outreach,''
rather than ``in-reach.''
We also found a diversity of approaches in the initial setup of single-subject
news outlets. Deep-Sea News and StartUp Beat are staffed by a single
founding editor and fueled by volunteer contributors. Two of the publishers
in our study, OpenCanada and FactCheck.org, were parented by and
embedded within think tanks. Their respective institutions, the Canadian
International Council and the Annenberg Public Policy Center, provided
grant funding and goods in kind, such as office space and IT support.
Other publishers have come to operate within a network model with affiliated
sister sites, a fusion of hyperlocal and hypertopical beats. Homicide
Watch Chicago was founded as a sister site to Homicide Watch websites in
Washington, D.C. and Trenton, N.J. Education News Network (Chalkbeat)
formed through a partnership between niche news sites Gotham Schools
and EdNews Colorado. Their respective founders are soon launching affiliates
in Tennessee and Indiana. Bleacher Report is a network of hypertopical
news beats focused on specific sports teams.
``It’s hundreds of individual niches. It’s the Vancouver Canucks or Michigan
State basketball, and we are an amalgamation of all those,'' said Bleacher
Report’s King Kaufman, Write-Program Director.
Tehran Bureau, which started as a self-funded independent news site, was
subsequently adopted by PBS FRONTLINE. In January 2013, founder
Kelly Golnoush Niknejad moved the site to The Guardian, which now
contributes fundraising support and goods in kind to boost its operations.
SoccerDrugs (Fussball Doping) similarly grew out of the investigative
unit of a German newspaper, Westdeutsche Allgemeine Zeitung. It


then moved on to independent operations when the founding publisher,
Daniel Drepper, enrolled as a student at Columbia University’s Graduate
School of Journalism.
Homicide Watch Chicago was parented by the Chicago Sun Times; Managing
Editor Craig Newman says they were given the space, resources
and autonomy to functionally run as an internal startup. He describes the
model, which works as a de facto return to beat reporting within the newspaper
ecosystem:
These [Homicide Watch] folks are in the newsroom, but they’re not
quite part of a newsroom structure…there’s a lot more flexibility
there. It has the ability to focus really narrowly on one content area,
one problem, one issue, which in this case, it’s a broad problem, but
it is a very narrow focus when you get down to it. This person was
killed, and here’s what it means to the people around him.
It’s essentially a hyper-narrow beat, and they don’t get called away
onto other nonsense, or they don’t get sucked into tangent conversations
about crime as a whole in the city. It’s putting a lens on this one
very specific problem.
Single-subject news startups have also been wholly acquired by mainstream
media partners. Bleacher Report was purchased by Turner Sports in August
2012, roughly six years after it went live. Capital New York, a single-subject
website not currently included in our research group (due to an initial 20
website cap for the preliminary study), was acquired by POLITICO in September
2013.
In at least one case, a mainstream media acquisition ended in divorce.
TreeHugger, a website dedicated to issues of sustainability and design, was
bought by the Discovery Networks in 2007 for a reported \$10 million. After
years of disharmonious operations, it was subsequently given away for free

to the Mother Nature Network. Lloyd Alter, the website’s managing editor,
said that TreeHugger fared better as an independent news site than it did
under Discovery, even from a financial perspective. He recounted:
The marriage never worked out. The TV people never understood
the digital people, and vice versa. The TV people couldn’t figure
out how to sell ads against it. As soon as they went away and we
could sell Google ads and others, without paying for Discovery’s
massive overhead, we were doing great. Suddenly we’re on a breakeven
basis. It’s pretty much through ads alone. We’re a nimble,
standalone business.
The Discovery experience hurt us in the long run. Everyone was
miserable, the editor had to answer to them all day. We were overwhelmed
by all-hands-on-deck meetings at Discovery. There was
just no fit.


\chapter{VI. Consistent Results: Emphasis on Engagement and Impact}
Despite the scattered limited metrics that publishers were able to provide—
many did not keep a consistent record—we did see a consistent pattern.
Key performance indicators appeared to reflect the consumption habits of
a niche community with high engagement from regular return visits from
what we’ve called the ``super users of the story,'' whom we define as news
consumers who actively and passionately seek niche news on their topic of
interest. The peak return rate among study respondents was 60 percent; the
mean was 39 percent.
In terms of traffic drivers, participants pointed to major news events within
their domains. For instance, elections and political debates drive traffic to
FactCheck.org and Politic365, while geopolitical headlines drive traffic to
Tehran Bureau and North Korea News. Those event drivers are often unpredictable,
but do work to the advantage of niche publications, boosting not
just their traffic but also their brand visibility and business development.
Between the peaks, publishers work to maintain momentum with core
audience members. StartUp Beat has seen an effective boost from guest
columns written by contributors with advanced personal followings. Both

OpenCanada and Gotham Gazette say that in-depth feature reports bring
high-value users to their sites. Deep-Sea News and TreeHugger see relative
spikes from stories about wildlife, with photos of cute and quirky animals.
Over time, publishers have come to identify which content may go viral,
and which traffic comes in at a higher value. Chad O’Carroll, Director and
Managing Editor of North Korea News explained:
We identified the kind of cigarettes that Kim Jung Un smokes, that
went viral. But the kind of pieces that attract subscribers are tailored
to North Korea watchers, students, policymakers.
Yet as a broad consensus, single-subject publishers focus less on traffic
volume than do traditional mainstream publications. Our respondents
instead place more emphasis on their missions, measuring success not
in traffic but in demonstrable impact as defined by the quality coverage
and public awareness of their topics. Kelly Virella, founder of The Urban
Thinker, described it as putting an emphasis on building ``a high-impact
journalism brand.''
That choice of optimizing for impact satisfies both the mandate and model
of most sites. David Sassoon at InsideClimate News sees it as key to sustaining
philanthropic support.
We’re not showing traffic numbers as evidence of our impact. We’re
pointing to things like our Reuters collaboration and we developed
others as we went along to show how we were impacting the conversation
[on climate change].
We found a way to have an impact on the conversation without having
to build a giant website with huge traffic. It’s very expensive and
difficult to do. So we found high-impact, low-cost solutions for doing
our work. That was attractive to funders.


Deep-Sea News, which is run by educators, puts an emphasis on knowledge
transfer: It aims to help a broader audience gain insight into ocean life. This
directive has afforded them a highly collaborative method and interaction
with sites that could otherwise be considered competitors; instead of competing,
they cross-promote. It’s a view shared across a handful of publishers
in our study.
``We don’t really work against anybody. The more people are talking about
the oceans, the better for us,'' said founder Craig McClain.
As an indicator of impact, roughly half of our respondents posted their work
to mainstream media outlets, including the cross-posting of articles written
by single-subject staff writers. HealthMap has collaborated with the professional
press, sharing its findings with The New England Journal of Medicine.
HealthMap also found itself documenting flu outbreaks earlier this year,
at a time when the Centers for Disease Control and Prevention was out of
operation during a federal government shutdown.
Another mark of influence has been imitation, or an apparent copycat influence
on the wider mainstream and startup press. Bleacher Report and Fact-
Check.org both saw this happen, with palpable results. Bleacher Report’s
King Kaufman explained:
The industry is following us around. This season, for the first time
ever, ESPN has a reporter assigned to every single NFL team. So
they have 32 beat writers, and they never had that before. And that’s
something we’ve had since the beginning…That’s the innovation.
FactCheck.org was founded with a single, guiding methodology: to take
the place of focused political fact-checking desks that used to be part of
newsroom operations but were downsized in cost-cutting moves. With the
proven popularity of its site, Director Emeritus Brooks Jackson has seen a
resurgence of the craft, with efforts like PolitiFact, a project of the Tampa
Bay Times, and The Fact Checker, a microsite within the Washington Post.

Brooks Jackson said:
Over the years, what I find most gratifying is the extent to which
we’ve been imitated and inspired others. I used to say, whenever I
was interviewed about this, that the mainstream news organizations
ought to be embarrassed that it’s been left to a little Ivy League think
tank [Annenberg Public Policy Center] to do work like this. I had
always considered it sort of the core First Amendment responsibility.


\chapter{VII. Key Challenges from Bandwidth to Business Management}
For single-subject news outlets, resourcing is the root of all challenges. As
a symptom, a big constraint is limited team bandwidth, especially in the
case of websites that are self-funded and reliant on volunteer labor. To succeed,
a news startup needs the same skill sets as any other startup—a mix of
technology, digital marketing and business strategy, all within the context
of lean operations.
One common hurdle our participants reported was the difficulty they
encounter in managing technology and iterative design. Publishers complained
about shoddy work from outsourced developers, a slow progression
of design aesthetics and frustration with content management systems. Brian
Kovalesky, Founder and Editor of StartupBeat, summed up the frustration:
Technology for me is always the biggest obstacle. As I went along, I
had to learn everything by myself. For all the talk about citizen journalism,
it’s really hard. Something always comes out looking funny,
or something doesn’t work.
To date, almost all of the respondents use an ad hoc marketing and user
acquisition strategy. Beyond maintaining social media feeds, with varying
levels of consistency, publishers are only able to devote occasional energy

toward visibility and community-building activities. Often these limited
marketing efforts result from a combined problem of limited know-how
and resource constraints.
One exception is publisher Kelly Virella at The Urban Thinker, a digital
magazine about black culture that is in its prelaunch phase. Virella has used
the time to develop a business plan and marketing strategy based on interviews
with target subscribers. She explained her approach:
I know that in order to succeed, you need a great marketing plan and
you’ve got to execute it to a T, and you have to have low user acquisition
costs…it’s all got to work out mathematically.
I’m really committed to the customer development process, and I’m
very committed to spending small amounts of money and seeing
what I can learn about what I’m doing and then iterating. But I really
did come to understand that there’s no way for us to make it short of
having the capital necessary to fund the endeavor, because we’d have
to hire a lot of people to bring our traffic levels to a point where we
could actually monetize the site.
Other publishers have used innovative, more improvised moves. FactCheck.
org announced its launch through a postcard press release to Washington,
D.C. journalists, while HealthMap pegged an awareness campaign to
the release of the film Contagion. Education News Network (Chalkbeat) is
hiring a dedicated director for engagement and growth, focused on digital
marketing. OpenCanada has taken an approach of direct marketing at academic
and industry events, as well as implementing a digital marketing strategy
using a Google AdWords grant. Few sites are currently doing advanced
search engine optimization or aggressive user-acquisition techniques.
One common marketing and distribution practice includes cross-posting
arrangements with major news outlets. OpenCanada, SoccerDrugs (Fussball
Doping) and InsideClimate News have freely shared content with


mainstream media partners, which has helped to drive traffic and enhance
brand awareness. Cameron Talk and Taylor Owen of OpenCanada called its
cross-posts in The Globe and Mail ``a milestone for credibility.''
All respondents are using some form of social media outreach, and some
are using well-developed social media strategies. Since its founding in 2005,
Deep-Sea News has learned to segment, leverage and deploy its social
media audience around a varied set of objectives. Founder Craig McClain
described the approach: ``We have a huge Facebook and Twitter following.
We have a specific hashtag that we use online that other people post to. We
have Pinterest and Tumblr.''3 He described how the advent of social media
also solved a problem around third-party content, allowing him to curate or
aggregate external content without featuring it on his home site:
As time went on and as Twitter and things like that developed, those
became better ways to draw people’s attention to content that didn’t
originate with us. Not only that, but we get much different audiences.
For example, if you take Pinterest, the audience there is primarily
female as opposed to the male audiences that we’ve tapped into
through some other outlets. So we treat them both as a mechanism
to drive traffic to the main site—the blog—but we also use that as a
mechanism to disperse content out that we don’t distribute through
the main blog.
For example, a lot of people will forward us announcements that they
want to send out to the rest of the community. We don’t put that on
the main site anymore…that main site is only for us to contribute
longer format original writing.
3 Pintrest demographics are proven to be predominantly female, based upon several sources’
estimates, including a study released in November 2012 by Insights in Marketing LLC.
``Marketing to Women—How Effective Are We at Connecting to Her?'' Insights in Marketing.
Insights in Marketing, LLC. 29 Nov. 2012. Web. 28 Oct. 2013.

Long-term goals for each publisher are fairly consistent: to grow operations,
expand coverage and pay writers and editors who currently work pro bono.
But most respondents recognized a challenge in organizing their strategic
planning and business development capacity. One publication, which had no
development roadmap, said its five-year plan was to ``live to die another day.''
In some cases the stories themselves make the enterprise harder to plan, as
with North Korea News. Founder Chad O’Carroll explained:
It’s really hard to plan on North Korea because unlike all the other
issues we have literally the most closed media environment, unless
you have an Eritrea news service…we can plan assuming the country
stays closed, or plan that the country is going to open up. It’s a really
tough one.
Only one publisher, Daniel Drepper of SoccerDrugs (Fussball Doping),
has built his website for planned obsolescence—a news outlet that’s akin
to a ``pop-up store,'' covering a single story. SoccerDrugs (Fussball Doping)
expects to shrink once it has a major breaking news story—successfully
implanting the issue of doping in soccer within mainstream press coverage.


\chapter{VIII. The Emerging Business Model: A Search for Alternate Revenue Streams}
We have seen a mix of traditional and innovative approaches to sustainable
revenue. Nearly half of the founding publishers in our study launched their
operations by self-funding all or part of their initial costs. Five publishers
had the help of foundation grants; four others were parented by think tanks
or given support from a local news outlet, generally in the form of office
space and goods in kind.
Those who relied on foundation support voiced a need to shift away from
grant dependency. In a two-year business plan migration, InsideClimate
News plans to move from a heavily foundation-subsidized model to a
diversified revenue stream that includes corporate sponsorships and an
individual donor network. Elizabeth Green at Education News Network
(Chalkbeat) hopes for a similar migration, with foundation support acting
as the kickstarter toward sustainable revenue. She explained the dynamic:
We think that the best way to get the startup capital to do this is from
philanthropy. But over time, we really believe that we can grow substantially
on [our] earned revenue.
There’s actually effectively an industry around educational change.
So, there’s actually quite a lot of opportunity for earned revenue as
well as philanthropy, philanthropic investment.

Traditional Advertising vs. Sponsorship
Many respondents felt that their baseline traffic volume was insufficient
to generate a meaningful revenue stream from traditional advertising. The
alternative approach of direct sponsorship seems more financially feasible;
single-subject publishers can deliver a targeted audience of key influencers
within their niche and leverage a rising trend of content marketing within
the advertising realm. But that approach raises an ethical question that publishers
within our study have yet to resolve: how to build a direct sponsorship
model that does not create a conflict of interest. The wrong sponsor or
the perception of an overly cozy relationship risks polluting the brand and
its perception of neutrality. For that reason the founders of StartUp Beat
and FactCheck.org have ruled out sponsored content completely.
All of these things that we do to maintain our integrity…kind of put a crimp
in the business model,” said Eugene Kiely, Director of FactCheck.org, which
does not take sponsorship and discloses all donations above \$1,000.
News Outlets Experimenting with Live Events
A number of respondents said they are considering or currently engaging
live events as an alternate revenue stream. TechPresident was founded
with a legacy of live events run by its parent entity, the Personal Democracy
Forum. Co-Founder Micah Sifry sees events as a way to build on the particular
advantage of single-subject websites that serve ``a community convening
role.'' Archinect’s original revenue model leaned heavily on live events. Similarly,
The Urban Thinker plans to hold salons and content-driven events as
part of its business model, bridging sponsors with its target audience. Kelly
Virella explained how she sees a level playing field for startup publishers
within that domain:


The nice thing about doing events is that we can have events that are
as large as any other large magazine can have. And so, in that sense,
we can compete, as long as we’re bringing together people that sponsors
would want to connect with.
StartUp Beat has explicitly avoided live events, concerned that event sponsorship
will yield a potential conflict of interest. Others see live events as
consistent with their mission, even enhancing their capacity to deliver on
it. Craig Newman at Homicide Watch Chicago is considering that route,
sponsoring open discussions with community and business leaders to ``foster
a more rapid and regular discussion'' on violent crime. That was a shared
view at Politic365. Founder Kristal High Taylor hosts quarterly events with
elected officials to deliver on its mandate of addressing ``an underrepresentation
of minorities in the national conversation.''
Marketplaces, Paywalls and User Donations
Merchandizing and marketplace functions have taken root at some publications.
Education News Network (Chalkbeat) runs a job board, fueled by
the high competition for teaching talent. North Korea News sells branded
merchandise on its website. StartUp Beat has considered running a marketplace
for buyers and sellers within its niche community, taking a commission
from sales.
Some publishers see paywalls and subscription models as antithetical to
their mission, contrary to the cause of spreading news and information
from their chosen field. One publisher in the study has established a subscription-
only service; IA Reporter produces its single-subject publication
as a professional-grade product, marketed to legal professionals who want
detailed reporting on cross-border arbitration. In this case, the content itself
makes the economics possible, given the high-stakes commercial value in
understanding and anticipating developments in the legal domain.

North Korea News runs a tiered-pricing, or ``free-mium,'' model: 5 free
articles per month for casual users, unlimited access for \$15 per month,
and premium, professional-grade content for \$100 per month. In the context
of this publication, the high-end consumer product consists of detailed
reporting on North Korean state media and senior military leaders.
Chad O’Carroll said it was a tumultuous shift from running free content to
landing on a sustainable subscription model.
Uptake was very slow at the start…a lot of people had gotten used to
this free and available service, so they were shocked when we put up
the paywall. The high pricing scared a lot of people off. Now we realized
that we need at least two people doing sales and screen-sharing
demonstrations. It’s making a real difference [and paying for itself].
Still, he said that predictable and sustainable revenue remains elusive.
It’s very difficult. It’s hand to mouth, taking each month as it comes.
The outgoings are high and incoming has been so erratic. Some
months there’s \$10,000 coming in, other months there’s \$1,000 coming
in. Now we’ve got this sales effort in business-to-business calls,
then we’ll be able to better plan.
Efforts to syndicate content in exchange for revenue have been limited, in
part because publishers have preferred the visibility and distribution value
in giving away their content for free. In an alternative approach, writers at
SoccerDrugs (Fussball Doping) have taken on occasional paid assignments
from mainstream outlets; freelance assignments from major newspapers
effectively draw on the staff’s expertise and contribute to the organization’s
budget. Repackaged content in the form of e-books is in its early days
as a revenue experiment, though InsideClimate News has published two
e-books and expects to publish at least two more by the end of the year.


Several publishers are currently soliciting community donations, while others
are considering this move. FactCheck.org, which gets the majority of its
funding from foundation grants, has succeeded in raising roughly \$70,000
per year in user contributions.

\chapter{IX. Data Analysis}
\section{Methodology}
To determine patterns and trends among our study participants, the
research team asked for quantitative data in a standardized questionnaire.
In this data form, the participants were asked for data points around traffic,
funding and budget expenditures as an organization. Specifically, all
participants were asked for traffic numbers in terms of hits per month
on a month-to-month basis for their launch year and their current year
(2013). For funding, participants were asked to identify the stream from
which their current funding comes and were provided the following categories:
seed funding/VC funding, individual donor contributions, grant
revenue, earned revenue/external projects, self-funding, subscription
revenue, and advertising revenue. This data was recorded by percentage
rather than by a specific number to protect company-sensitive information.
Lastly, participants were asked to provide their budget expenditures
from this year in the following categories: personnel salaries, technology
development, operations, marketing, content development, and external
projects. This information was also recorded by percentage to protect
company-sensitive information.
Based on this information, the research team was able to identify trends and
patterns in traffic, revenue streams, and areas of spending per each organization
by identifying commonalities across data sets.


\section{Troubleshooting}
One of the greatest areas of ambiguity when interpreting the data resulted
from some companies’ lack of metrics-gathering tools, such as Google Analytics
or other built-in metrics systems, during their launch year. Also, certain
participants either did not choose to provide or did not have access to
certain data points within the questionnaire, making the data difficult to
measure as a whole.
An even greater problem lay with our researchers’ inability to contextualize
the information present in the data sets according to the companies’ year of
launch and the thematic issues its site covers. However, revealing the information
would make it too easy to identify each participant. Out of consideration
for their privacy, the study was not designed to reveal financial and
technical data about the companies participating.
Yet, from the data that is public, one can glean a lot of important lessons
from our participants about company behavioral patterns, traffic patterns,
funding and expenditures that are useful in identifying best practices and
areas for improvement among single-subject websites across the board.
Below are the key findings from each level of analysis, including: funding,
budget expenditures and traffic data.
Site Traffic
In developing the measurements for this section, the research team ran up
against two areas of issue: first, the lack of analytics tools used in participants’
launch years, and, second, displaying information based upon site
traffic numbers.

Since, across the board, most participants failed to collect site analytics in
their reported launch year, the research team has just listed 2013 traffic
numbers, rather than both launch year and 2013, so as not to skew data
based on varying start dates and months of each organization.
To develop the parameters for Chart A.1, the researchers took the net average
of each website’s traffic in terms of hits per month and then divided each
category into two tiers based on the range of numbers reported. Because
the top tier’s range was significantly larger than the most common tier, the
team chose not to include those outliers.


Looking at the data collectively, steady daily traffic appears as driven by an
average of 37 percent user return rate (individual responses not displayed).
Spikes in traffic are influenced by events most likely related to the release
of a particular article or an increase in attention paid to the niche subject
in popular culture, business, political events or mainstream media. Dips in
traffic, while harder to explain, are influenced by a drop in interest or knowledge
of the niche topic, consistent with varying attention spans for specific
topics in mainstream media and the general media-consuming population.
In analyzing the data within the most common tier, it’s clear that the majority
of sites’ traffic lies in the 0-90,000 hits-per-month range. Considering the
niche specialty of the topic areas from our participant pool, this expectant
range of site traffic can be applied to project traffic to other single-subject
website. This data is important for advertising and subscription opportunities,
which are limited by an audience size of 0-90,000 for specialized topics,
excluding outlier responses. With this expected visitor base, outside funding
must be considered to adequately cover organizational expenses.

\section{Funding}
The data presented in Chart A.2 shows the sources from which the participants
draw their revenue. In the past, newspaper revenue relied heavily
upon the sale of advertising and subscription fees, and minimally upon
other sources of revenue like events or grant funding. Traditionally subscription
fees did not subsidize the production of the news, but rather the
cost of printing and distributing the newspaper (which is now mitigated via
an online distribution model that costs significantly less, if anything). This
presents an interesting point of analysis, looking at the ability of an online
niche publication to attract subscription-based revenue.
What we found in this study is that the most common revenue source
respondents identified was grants (9), then individual donors (7) and advertising
(6), followed by self-funding and subscription revenue (5) as well as

external projects (4). On the lower end of responses was seed-funding. Not
surprisingly, most who identified grant revenue also identified receiving
individual donor contributions (7 out of 9). While many listed grant funding,
only four respondents included subscriptions as part of their revenue
stream, and only two out of those four offered it as a significant portion
of their revenue. While clearly advertising and subscription still play an
active role in funding for these sites, the data favors grant- or donationbased
funding.

In addition to the data collected, our research team asked participants to
identify their ideal amount of funding for one year of operations to better
establish the range of funding necessary to operate a single-subject website.
While two participants’ ideal budget ranged into the millions, and one
quoted just \$20,000, the most common response was \$500,000 for one year
of operations. The average, excluding the aforementioned outliers, was
\$358,917, which is highly divergent from traditional newsrooms, which can
cost millions of dollars per year.
Budget Expenditures
The data in Chart A.3 demonstrates that the majority of expenses participants
reported fall under the categories of operations (9) and personnel (9)
then technology development (8), followed by content development (6).
Only three respondents listed external projects as a budget expenditure this
year. While many of the businesses place emphasis on operations and personnel
as funding priorities, responses indicated that less important funding
priorities include marketing and external projects. Respondents noted
that content development and operational personnel salaries are often
interlinked, since many employee pools consist of journalists or reporters
fulfilling different operational, administrative, and marketing roles in addition
to their content development roles at their organizations.

Chart A.3 – Budget Expenditures Analysis
The participant interview process yielded one consistent finding: the lack
of formalized marketing programs. The data shows this trend as constant,
with only five respondents allocating any percentage of their budget toward
marketing, and only one dedicating a large chunk of its expenditures to
marketing. The research team discovered that while many journalists can
identify with the concept of working to ``get eyes on a story,'' very few identified
this process as marketing. In fact, during the interview process many
participants expressed a desire to see how other companies conducted their
marketing and internal operational procedures. As a qualitative assessment,
it is fair to conclude that many journalists were not formally trained in business
operations, especially among older generations. As the newsroom
paradigm shifts toward one wherein online journalists are now expected to
market their websites as a product, this is an education gap that may need
to be addressed in the future.



\chapter{X. Early Hypotheses/Key Takeaways}
We expect to see single-subject websites persist and proliferate, largely
on the back of foundation funding and personal sacrifice by their founding
teams. But to thrive they will need a fusion of reporting experience and
startup savvy—essentially, the skills of small business management. News
entrepreneurs, like any other entrepreneurs, need a combination of prowess,
persistence and passion. Paul Petrunia, founder of Archinect, commented:
When I see a new startup like a blog, it’s immediately apparent if it’s
being driven by a passion or not. When it’s not, I have very little faith
it will last. So I think that passion is incredibly important, especially
in the beginning, when sometimes that’s really the only thing that’s
keeping you going. Every site that I’ve seen that’s done well has really
been narrowly focused. I think that’s what people expect and look for
these days.
We anticipate single-subject news will accelerate the trend of ``unbundling''
the newsroom: Bleacher Report as a spun-off sports desk, FactCheck.org
as the outsourced function of a political desk, Education News Network
(Chalkbeat) as a specialized bureau on local schools.
Today’s generation of journalists can and will create new content where they
see a deficit. The demand for in-depth news coverage by niche audiences is
spawning new salable products for the digital marketplace. The unanswered
question remains as to where supply and demand will level off, thus shaping
the emerging field of news startups into a stable ecosystem of quality work.